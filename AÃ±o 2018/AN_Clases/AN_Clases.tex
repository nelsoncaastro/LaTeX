
\documentclass[12pt]{article}
\title{Clases de Analisis Numerico en \LaTeX}
\author{Nelson Castro}
\begin{document}
\date{2018}
\maketitle
\newpage
\section*{Analisis Numerico Clase (16/5)}
\subsection*{Conceptos sobre el Espacio Euclideano}
Norma Euclideana de un vector en {\large $R^n$}
\linebreak

Considere el vector	{\large $\left[
\begin{array}{c}
X_1 \\
X_2\\
X_3 
\end{array}
\right] \in R^3$ } 

Cuyo tamaño modulo o norma es:

{\large $\sqrt{{X_1}^2+{X_2}^2+{X_3}^2}=||\overline{X}||$}

Sea {\large $\overline{Y} = \left[
\begin{array}{c}
Y_1 \\
Y_2\\
Y_3\\
Y_4\\
Y_5 
\end{array}
\right] \in R^5$ }

{\large $\sqrt{{Y_1}^2+{Y_2}^2+ \cdots + {Y_5}^2}=||\overline{Y}||$}

En general sea el vector
{\large $ \overline X = \left[
\begin{array}{c}
X_1 \\
X_2\\
X_3 \\
. \\
. \\
. \\
X_n
\end{array}
\right] \in R^n$ }

Donde la norma euclideana es
{\large $||X||=(\sum_{i=1}^{n}{X_i^2})^\frac{1}{2}$}

Propiedades de la norma en el espacio {\large $R^n$}
\begin{itemize}
\item

\item
\end{itemize}

Propiedades del producto interno
\begin{itemize}
\item
{\large $(\overline{X},\overline{Y})=(\overline{Y},\overline{X})  \forall \overline{X}, \overline{Y} \in R^n$}
\item
{\large $(\overline{X},\overline{X}) \ge 0 ; (\overline{X},\overline{X})= \sum_{i=1}^{n}{(X_i)^2}$}
\item

\end{itemize}

Ortogonalidad entre vectores de {\large $R^n$} 

Sean {\large $\overline{X}, \overline{Y} \in R^n$}, para los cuales se establece: $(\overline{X} y \overline{Y})$ son ortogonales si $(\overline{X},\overline{Y})=0$

Notar que el vector nulo $\overline{0}$ es ortogonal o cualquier otro vector

Un sistema de vectores $ \overline{X_1},\overline{X_2},\overline{X_3},\cdots,\overline{X_n}$ con $\underline{X}_i \ne 0$

\newpage

\begin{table}[!th]
\begin{center}
\begin{tabular}{|c|c|}
\hline
ISO: & HDLC (Control de enlace de datos de alto nivel) \\
IEEE: & 802.2 (LLC) \\
  & 802.3 (Ethernet) \\
  &	802.5 (Token Ring) \\
  & 802.11 (Wireless LAN) \\
ITU: & Q.922 (Estandar de Frame Relay)\\
  & Q.921 (Estandar de enlace de datos ISDN)\\
  & HDLC (Control de enlace de datos de alto nivel) \\
ANSI: & 3T9.5 \\
  & ADCCP (Protocolo de control de comunicación avanzada de datos) \\
\hline
\end{tabular}
\end{center}
\caption{Estandares para capa de enlace de datos}
\label{ex:tablon}
\end{table}

\newpage
\section*{Analisis Numerico Clase (21/5)}
\subsection*{Caracteristica de la solución al problema de los minimos cuadrados}

$\displaystyle F^* = C_0Q_0(x)+ C_1Q_1(x)+ C_2Q_2(x)+\cdots+ C_nQ_n(x)$

{\large $ \underline X = \left[
\begin{array}{c}
X_0 \\
X_1 \\
X_2\\
X_3 \\
. \\
. \\
. \\
X_n
\end{array}
\right]$ }
{\large $ \underline f = \left[
\begin{array}{c}
f(X_0)\\
f(X_1) \\
f(X_2)\\
f(X_3) \\
. \\
. \\
. \\
f(X_n)
\end{array}
\right]$}

{\large $ \underline Q_1 = \left[
\begin{array}{c}
X_0\
X_1 \\
X_2\\
X_3 \\
. \\
. \\
. \\
X_n
\end{array}
\right]$}
{\large $ \underline Q_0= \left[
\begin{array}{c}
1\\
1 \\
1\\
1 \\
. \\
. \\
. \\
1
\end{array}
\right]$}

Se desea: {\large $ \underline f - \underline{f^*} = \left[
\begin{array}{c}
|f_0-f^*_0| \\
|f_1-f^*_1| \\
|f_2-f^*_2|\\
|f_3-f^*_3| \\
. \\
. \\
. \\
|f_n-f^*_n|
\end{array}
\right]$}
$||f_0-f^*_0||^2 \textbf{ sea lo menos posible}$

$\displaystyle \underline F^* = C_0 \underline Q_0(x)+ C_1 \underline Q_1(x)+ C_2 \underline Q_2(x)+\cdots+ C_n \underline Q_n(x)=\sum_{i=0}^{n}{a_iQ_i}$

$\displaystyle F^*$ es tal que $\displaystyle \underline {f-f^*}$ es ortogonal al espacio de $\displaystyle Q_0, Q_1,\cdots,Q_n$

$\displaystyle \underline{f}- \sum_{i=0}^{n}{a_iQ_i}=(f^*-\sum_{i=0}^{n}{a_iQ_i})+(f- \underline{f^*})$

$\displaystyle \underline{f}- \sum_{i=0}^{n}{a_i \underline Q_i}=(\sum_{i=0}^{n}{c_iQ_i}-\sum_{i=0}^{n}{a_iQ_i})+(f- \underline{f^*})$

\newpage

\section*{Analisis Numerico Clase (23/5)}

$\displaystyle f(x)\textit{  en  } [a,b]$

$\displaystyle (f,q)=\int_{a}^{b}{f(x)\cdot g(x)\cdot w(x) dx}$, \hspace{1cm}$\displaystyle w(x)=1\textit{  en  } [a,b]$

$\displaystyle (f,q)=\int_{a}^{b}{f(x)\cdot dx}$

$\displaystyle (f,f)=\int_{a}^{b}{f^2(x)\cdot g(x)\cdot dx}$

$\displaystyle ||f||=\sqrt{\int_{a}^{b}{f^2(x)\cdot dx}}$

Ejemplo: 

$\displaystyle f(x)=e^{-x}\textit{  en  } [0,1]$

$\displaystyle P(x)=c_0+c_1x+c_2x^2$, con $\displaystyle Q_0(x)=1,Q_1(x)=x, Q_2(x)=x^2$


$\displaystyle \begin{array}{c}
C_0(Q_0,Q_0)+C_1(Q_1,Q_0)+C_2(Q_2,Q_0) = (f_1Q_0)\\
C_0(Q_0,Q_1)+C_1(Q_1,Q_1)+C_2(Q_2,Q_1) = (f_1Q_1)\\
C_0(Q_0,Q_2)+C_1(Q_1,Q_2)+C_2(Q_2,Q_2) = (f_1Q_2)\\
\end{array}$


$\displaystyle \begin{array}{c}
C_0\int_{0}^{1}{1\cdot dx}+C_1\int_{0}^{1}{x\cdot dx}+C_2\int_{0}^{1}{x^2\cdot dx} = \int_{0}^{1}{e^{-x}(1)dx}\\
C_0\int_{0}^{1}{x\cdot dx}+C_1\int_{0}^{1}{x^2\cdot dx}+C_2\int_{0}^{1}{x^3 \cdot dx} = \int_{0}^{1}{e^{-x}(x)dx}\\
C_0\int_{0}^{1}{x^2\cdot dx}+C_1\int_{0}^{1}{x^3\cdot dx}+C_2\int_{0}^{1}{x^4\cdot dx} = \int_{0}^{1}{e^{-x}(x^2)dx}\\
\end{array}$

$\displaystyle \begin{array}{c}
C_0=0.994489\\
C_1=-0.930548\\
C_2=0.308719\
\end{array}$

$\displaystyle \underline{Q}=[Q_0,Q_1\cdots , Q_n]$

$\displaystyle \underline{Q^t} = \left[
\begin{array}{c}
Q_0 \\
Q_1 \\ 
. \\
.\\
.\\
Q_n
\end{array}
\right]; \underline{A} \cdot \underline{X}= \underline{B}$

$\displaystyle \underline{X} =\underline{A^{-1}}\cdot \underline{B}$

$\displaystyle \underline{A}=\int_{a}^{b}{\underline{Q^t}\cdot \underline{Q} \cdot dx}$

$\displaystyle \underline{B}=\int_{a}^{b}{\underline{Q^t}\cdot f(x) \cdot dx}$

\newpage
\section*{Analisis Numerico 4/6}
\subsection*{Las diferencias finitas}

\underline{Conceptos:}

Se denota por $Y_n$ a la sucesión \{$Y_n$\} las operaciones de traslación "E" y de diferencias progresivas "$\Delta$" se definen como: 
$EY_n=$
esto es una prueba que el lint sigue existiendo en este archivo y ni puta idea que carajo es $h_{ola}$
\subsection{Puro vacilongo delicioso}
  Que delicia esta mierda que funciona super delicioso para todo
  $Y_1$
  $Este es el gran vacil^3$
  Yo se lo que le estoy diciendo lito dog
  Hola mi copo drokcs
  \subsection{Ecuaciones diferenciales.}
  Variable dependiente
  \\ Variable independiente, ctes
  \\ Derivadas (ordinarias o parciales) de la variable dependiente
  \\$F(x,y^I,y^{II}, \cdots y^{(n)},c)=0$
  \\E.D.Ordinaria de orden "n"
  \\Grado es el exponente de la mayor potencia
  \\$2xyy^I-ex^2(y^I)^2=0$
  \\E.D Ordinaria de primer orden y segundo grado
  \\$2\frac{dy}{dt}-t^2 \left(
 \frac{d^2}{dt^2}   
  \right)^2=4\cdot Cos(t)$
  \\E.D Parcial de Segundo orden de primer grado
  \\Resolver una E.D implica hallar una función $y=f(x)+c$ ó $y=f(t)+c$
  \\que al sustituir en la E.D de primer orden, la satisface
  \\$y=2e^{-2x}+\frac{1}{3}e^x$
  sol de $y^I+2y=e^x$
  \\Ejemplo
 \\ $y=1+c\sqrt{1-x^2}; (1-x^2)y^1+xy=x$

  \subsection{Ecuaciones Diferenciales de primer orden}
  \begin{itemize}
    \item E.D Variables separables
    \item E.D Homogeneas
    \item E.D de Coeficientes Lineales
    \item E.D Exacta
    \item E.D por factor integrante
    \item E.D lineal de primer orden
  \end{itemize}
  \subsection{E.D Variables separables}
  $F(X,Y,Y^I,c)=0$
  \\ $y^\prime=G(x,y)\rightleftharpoons\frac{dy}{dx}=G(x,y)$
  $y^\prime=\frac{\Phi(x)\mu(x)}{h(x)P(y)}$
  $\frac{dy}{dx}=\frac{\Phi(x)\mu(x)}{h(x)P(y)}$
  $M(x,y)dx - N(x,y)dy = 0$
  forma diferencial de la E.D
  $\Phi(x)\cdot\mu(y)dx-h(x)\cdot P(y)dy=0$
  $div\div h(x)\cdot\mu(y)$
  $\frac{\Phi(x)\mu(x)}{h(x)P(y)}dx-\frac{h(x)P(y)}{h(x)\mu (y)}dy=0$  
  $\frac{\Phi(x)}{y}$ aquí me quede :(
  \\Integrando:
  \\$\int\frac{\Phi(x)}{h(x)}dx-\int\frac{P(Y)}{=\mu(y)}dy=C$
  \\La constante C solo puede hallarse cuando se proporciona una condición inicial $y(0)=y_0\rightarrow (0,y_0)$
  \\o bien puede darse otro punto conocido $y(x_0)=y_0\rightarrow (x_0,y_0)$
  \\Ejemplo
  \\$xy\frac{dy}{dx}=1+y^2$ 
  \\$y^\prime=4e^{x+y}$ y luego encuentre la solución particular que pasa por el origen (0,0)
  \\$\frac{dy}{dx}=4e^x e^y$
  \\$\frac{dy}{e^y_0}=4e^x dx$
  \\$\int e^{-y}dy=\int 4 e^x dx$
  \\Integrando
  \\$-e^{-y}=4e^x +c$
  \\$(0,0)$
  \\$-1=4(1)+c$
  \\$c= -5$
 
  \newpage
  \section{Analisis Numerico 20/06}
\subsection{Ecuaciones Diferenciales Homogeneas(EDH)}
 Dice que la funcion F(X,y) es homogenea, si cumple cualquiera de las siguientes condiciones
\begin{itemize}
  \item $f(tx,ty)=t^n f(x,y)$
  \item $f(x,y)=x^n f(1,\frac{y}{x})$
  \item $f(x,y)=y^n f(\frac{x}{y},1)$
\end{itemize}
Siendo "n" grado de homogeneidad

Una E.D de la forma $M(x,y)dx+N(x,y)dy=0$ es homogenea cuando $M(x,y)$ y $N(x,y)$ son homogeneas del mismo grado.
\\ \textbf{Teorema}
\\Toda EDH se transforma en una EDVS al hacer la sustitucion $y=ux$ o $x=vy$
\\$y=ux \rightarrow dy=udx+xdu$
\\$x=vy \rightarrow dx=vdy+ydv$

\subsection{Ecuaciones Diferenciales Exactas (EDE)}
Es toda ecuacion diferencial de la forma $M(x,y)dx+N(x,y)=0$ en las que se cumple que $\frac{dN}{dx}=\frac{dM}{dy}$ siendo su solución $\Phi(x,y)+c=0$

\subsection{Ecuacion Diferenciales por Factor Integrante}
Considere la siguieinte E.D $(x+y)dx+(x\cdot ln(x))dy=0$
$M:x+y \Rightarrow \frac{dM}{dy}=1$
$N:x\cdot ln(x) \Rightarrow \frac{dN}{dx}=ln(x)+1$
$M(x,y)dx+N(x,y)dy=0$

\section{Clase de analisis numerico (4/7)}
Esta clase va a ser acerca de ecuaciones diferenciales y vamos a hablar del espacio euclididano.
\\ $\displaystyle \int _{-2}^{4}{x^2 \cdot \sqrt{ab+4dc} \cdot dx} = \frac{\sigma ^2 }{sen(y)}$
\newpage
Entonces las partes de espacio euclidiando son las siguientes:
\begin{itemize}
  \item Principio 
  \item Consecuente 
  \item Cuerpo 
  \item Analisis
  \item Conclusiones
  \item Fin
\end{itemize}
\end{document}

