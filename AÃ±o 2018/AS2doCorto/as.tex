\documentclass[12pt]{report}
\title{Analisis de Sistemas en \LaTeX}
\author{Nelson Castro}
\begin{document}
\maketitle
\newpage
\begin{center}
\section*{Método de Newton Modificado}
{\Large $\displaystyle P_{n+1}=P_n-\frac{f(P_n)}{\frac{f(P_n+h)-f(P_{n-1})}{(P_n+h)-P_{n-1}}}$}
\section*{Método $\Delta^2$ de Aitken}
{\Large \^P $\displaystyle =X_n-\frac{(X_{n+1}-X_n)^2}{X_{n+2}-2X_{n+1}+X_n}$}

\begin{itemize}
	\item
	Es un método de \textbf{extrapolación}, es decir, utiliza los estimaciones anteriores de \textit{b} para predecir una mejor aproximación \^ P.
	\item
	Tiene efectos en \textbf{cualquier sucesión linealmente convergente}.
	\item
	No necesita de evaluaciones adicionales de $g(x)$, simplemente trabaja sobre los que ya están calculados como cualquier interación de punto fijo.
\end{itemize}
{\huge $\displaystyle S_n=\sum_{k=0}^{n}{\frac{(-1)^k}{2k+1}}$}
\begin{table}[!th]
\begin{center}
\begin{tabular}{c|c|c|c|c|c}
\hline
\textit{n} & $X_n$ & $X_{n+1}$ & $X_{n+2}$ & \^P$_n$ & $E_{abs}$ \\
\hline
0 & $X_0$ & $X_1=g(X_0)$ & $X_2=g(X_1)$ & \^P$_0$ & $\not{\exists}$ \\
\hline
1 & $X_1$ & $X_2=g(X_1)$ & $X_3=g(X_2)$ & \^P$_1$ & $|$\^P$_1-$\^P$_0|$ \\
\hline
2 & $X_2$ & $X_3=g(X_2)$ & $X_4=g(X_3)$ & \^P$_2$ & $|$\^P$_2-$\^P$_1|$ \\
\hline
. & . & . & . & . & . \\
. & . & . & . & . & . \\
. & . & . & . & . & . \\
\hline
n & $X_n$ & $X_{n+1}=g(X_n)$ & $X_{n+2}=g(X_{n+1})$ & \^P$_n$ & $|$\^P$_n-$\^P$_{n-1}|$ \\
\hline
\end{tabular}
\end{center}
\caption{Método iterativo basado en una iteración de punto fijo $g(x)$}
\end{table}

\begin{itemize}
	\item
	Valores iniciales: $X_0$ (dado por el ejercicio)
	\item
	Respuesta: \^P$_n$
	\item
	Condición de paro: $E_{abs} \leq \epsilon$ (dado por el ejercicio)
\end{itemize}
\section*{Método de Steffensen}
\begin{itemize}
	\item
	El método de Steffensen está basado en el método $\Delta^2$ de Aitken.
	\item
	A diferencia del método de Aitken, esta vez \textbf{introducimos la aproximación} en el $g(x)$. Para así obtener un mejor estiamdo.
	\item
	\^P se considera un buen estimado para la raíz, por lo que introduce como valor inical $X_n$ en la proxima ronda de iteraciones.
	\item
	Puede demostrarse que el método de Steffensen exhibe un \textbf{orden de convergencia cuadrático}.
\end{itemize}

\begin{table}[!th]
\begin{center}
\begin{tabular}{c|c|c|c|c|c}
\hline
\textit{n} & $X_n$ & $X_{n+1}$ & $X_{n+2}$ & \^P$_n$ & $E_{abs}$ \\
\hline
0 & $X_0$ & $X_1=g(X_0)$ & $X_2=g(X_1)$ & \^P$_0$ & $\not{\exists}$ \\
\hline
1 & $X_1=$\^P$_0$ & $X_2=g(X_1)$ & $X_3=g(X_2)$ & \^P$_1$ & $|$\^P$_1-$\^P$_0|$ \\
\hline
2 & $X_2=$\^P$_1$ & $X_3=g(X_2)$ & $X_4=g(X_3)$ & \^P$_2$ & $|$\^P$_2-$\^P$_1|$ \\
\hline
. & . & . & . & . & . \\
. & . & . & . & . & . \\
. & . & . & . & . & . \\
\hline
n & $X_n=$\^P$_{n-1}$ & $X_{n+1}=g(X_n)$ & $X_{n+2}=g(X_{n+1})$ & \^P$_n$ & $|$\^P$_n-$\^P$_{n-1}|$ \\
\hline
\end{tabular}
\end{center}
\caption{Método iterativo basado en una iteración de punto fijo $g(x)$}
\end{table}

\begin{itemize}
	\item
	Valores iniciales: $X_0$ (dado por el ejercicio)
	\item
	Respuesta: \^P$_n$
	\item
	Condición de paro: $E_{abs} \leq \epsilon$ (dado por el ejercicio)
\end{itemize}

\section*{Método de Müller}
\begin{itemize}
	\item
	El método de Müller utiliza tres aproximaciones iniciales, $X_0,X_1,X_2$, para determinar la siguiente aproximación $X_3$ utilizando un método de extrapolación con un polinomio grado 2.
	\item
	La parábola se genera a partir de los 3 puntos iniciales, y puede predecir donde estará la raz de la función $f(x)$.
	\item
	El método de Müller suele ser bastante  efectivo para encontrar las \textbf{raíces complejas} de cualquier función.
\end{itemize}

$\displaystyle X_{n+3}=X_{n+2}-\frac{2c}{b+signo(b)\sqrt{b^2-4ac}}$

$\displaystyle X_{n+3}=X_{n+2}+ h$

$\displaystyle h_1=X_{n+1}-X_n$

$\displaystyle h_2=X_{n+2}-X_{n+1}$

$\displaystyle \delta_1=(f(X_{n+1})-f(X_{n}))/h_1$

$\displaystyle \delta_2=(f(X_{n+2})-f(X_{n+1}))/h_2$

$\displaystyle a=\frac{\delta_2-\delta_1}{h_2+h_1}$

$\displaystyle b=\delta_2 + h_2a$

$\displaystyle c=f(X_{n+2})$

$\displaystyle h=-\frac{2c}{b+signo(b)\sqrt{b^2-4ac}}$

$\displaystyle 
signo(b)= \left\{\begin{array}{lr}
	1 & $si  $ |b-\sqrt{b^2-4ac}| < |b+\sqrt{b^2-4ac} |  \\
	-1 & $si  $ |b-\sqrt{b^2-4ac}| \geq |b+\sqrt{b^2-4ac} | 
\end{array}\right\}
$

\begin{table}[!th]
\begin{center}
\begin{tabular}{c|c|c|c|c|c}
\hline
\textit{n} & $X_n$ & $X_{n+1}$ & $X_{n+2}$ & \^P$_n$ & $E_{abs}$ \\
\hline
0 & $X_0$ & $X_1=g(X_0)$ & $X_2=g(X_1)$ & \^P$_0$ & $\not{\exists}$ \\
\hline
1 & $X_1=$\^P$_0$ & $X_2=g(X_1)$ & $X_3=g(X_2)$ & \^P$_1$ & $|$\^P$_1-$\^P$_0|$ \\
\hline
2 & $X_2=$\^P$_1$ & $X_3=g(X_2)$ & $X_4=g(X_3)$ & \^P$_2$ & $|$\^P$_2-$\^P$_1|$ \\
\hline
. & . & . & . & . & . \\
. & . & . & . & . & . \\
. & . & . & . & . & . \\
\hline
n & $X_n=$\^P$_{n-1}$ & $X_{n+1}=g(X_n)$ & $X_{n+2}=g(X_{n+1})$ & \^P$_n$ & $|$\^P$_n-$\^P$_{n-1}|$ \\
\hline
\end{tabular}
\end{center}
\caption{Método iterativo basado en una iteración de punto fijo $g(x)$}
\end{table}

\end{center}
\end{document}