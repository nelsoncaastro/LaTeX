\documentclass[12pt]{report}
    \title{Corto 3 en \LaTeX}
    \author{Nelson Castro}
    \begin{document}
        \maketitle
        \newpage
        \subsection{Funciones continuas de Probabilidad}
        $f:\Re \Rightarrow \Re$ es función de probabilidad de $V.\delta$ continua, si cumple:
        \begin{itemize}
           \item  
           $f(x)>0,\forall x\in Df$
           \item
           $\int_{-\infty}^{\infty} f(x)dx=1$
        \end{itemize}
        Cual es la probabilidad que tome el valor de a?
        \\ $P(x=a)-\int_{a}^{b}f(x)dx=0$        
        \\ En las funciones continuas no hay probabilidades puntales ya que no habria area en un punto.
        \\ $f(a)\Rightarrow$ Punto de la curva en a (imagen de a)
        \\ \underline{La media}
        \\$\mu = \int_{-\infty}^{\infty}x\cdot f(x) dx$
        \\ \underline{La varianza}
        \\$\sigma^2=E(x^2)-\mu^2$
        \subsection{Funcion aumulada de probabilidad(F)}
        P($X\leq X_0$)$=\int_{-\infty}^{X_0}f(x)dt$
            \\Hola banchon
         \\$\sigma _{\overline{x}}=\frac{\sigma}{\sqrt{n}}$
         \\$Z=\frac{\overline{x}-\mu}{\sigma _{\overline{x}}}$
    \end{document}
