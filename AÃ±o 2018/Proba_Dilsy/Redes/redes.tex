\documentclass[12pt]{report}
    \title{Documentación proyecto Redes de Computadoras}
    \author{Nelson Giovanni Castro Rodas, \\ Henry Luis Banchon Gallardo, \\ Rodrigo Alejandro Alvarenga Perez, \\ Andres Antonio Quijada Muñoz}
    \begin{document}
        \maketitle

        \newpage
        \section*{Explicación infraestructura de red}
        Se hara uso del emulador VirtualBox, para simular una red que contenga lo siguiente:
        \begin{itemize}
            \item Cliente A
            \item Cliente B
            \item Servidor A
            \item Servidor B
        \end{itemize}
        \subsection*{Cliente A}
        Se utilizará para realizar pruebas de conexión y uso de servicios de red. Se ha decidido la utilización de Ubuntu Budgie 18.04.
        \subsection*{Cliente B}
        Se utilizará también para realizar pruebas de conexión y uso de servicios de red. Se ha decidido la utilización de Windows 7 Ultimate.
        \subsection*{Servidor A}
        Se ha designado como servidor principal, este servidor funcionará como servicio de DNS para la red interna, y además se le configurará el servicio de proxy-server (SQUID) que será responsable de interconectar la red interna de clientes (A y B) con otra red externa. 
        Se ha decidido la utilización de Ubuntu Server 18.04.
        \subsection*{Servidor B}
        Se ha designado como servidor interno, y se encargará como proveedor de los servicios DHCP y apache. Se ha decidido la utilización de Ubuntu Server.
        \section*{Vista general de los temas a tratar}
        \subsection*{Servidor A}
        \begin{itemize}
            \item Instalar dos tarjetas de red: una para red interna (estática) y una para red externa y lograr tener acceso a internet.
            \item Configurar el servicio de proxy-cache para dar acceso a internet al resto de host de la red interna. Bloquear páginas principales de entretenimiento.
            \item Configurar el servicio DNS primario y asignar nombres de dominio a los servidores A y B.
        \end{itemize}
        \subsection*{Servidor B}
        \begin{itemize}
            \item Configurar el servicio de DHCP para que los clientes puedan conseguir su información de red vía DHCP.
            \item Configurar el servicio de Apache Server como servidor interno. Instalar la plataforma Wordpress, para proveer a los clientes acceso a la misma.
        \end{itemize}
        \subsection*{Clientes}
        Realizar pruebas de los servicios: DNS, DHCP, acceso a Internet y Apache.
    \end{document}